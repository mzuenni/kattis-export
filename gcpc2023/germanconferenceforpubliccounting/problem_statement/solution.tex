\begin{frame}
  \frametitle{\problemtitle}
  \begin{block}{Problem}
    Count the number of signs with digits needed to display all numbers from $0$ to $n$.
  \end{block}
  \begin{block}{Example ($n=15$)}
    \begin{center}
      \fbox{1}\fbox{5}, \fbox{1}\fbox{4}, \fbox{1}\fbox{3}, \fbox{1}\fbox{2}, \fbox{1}\fbox{1}, \fbox{1}\fbox{0}, \fbox{9}, \fbox{8}, \fbox{7}, \fbox{6}, \fbox{5}, \fbox{4}, \fbox{3}, \fbox{2}, \fbox{1}, \fbox{0}
    \end{center}
    \vspace{-2mm}
    We need $11$ signs: \fbox{0}, \fbox{1}, \fbox{1}, \fbox{2}, \fbox{3}, \fbox{4}, \fbox{5}, \fbox{6}, \fbox{7}, \fbox{8}, \fbox{9}
  \end{block}
  \pause
  \begin{block}{Solution}
    \begin{itemize}
      \item<+-> For each digit, find the number in the range that uses the most copies of that digit.
      \item<+-> For each digit from $1$ to $9$:
        \begin{itemize}
          \item Find the longest \emph{repdigit} (number made up only of that digit) not exceeding $n$.
          \item Add its length to the result.
        \end{itemize}
      \item<+-> For the digit $0$:
        \begin{itemize}
          \item We always need at least one sign for the end of the countdown.
          \item The smallest number to use two signs is $100$, the smallest to use three signs is $1000$, \dots
          \item Find the largest power of $10$ not exceeding $n$ and add the appropriate number of $0$ signs.
        \end{itemize}
    \end{itemize}
  \end{block}
\end{frame}
