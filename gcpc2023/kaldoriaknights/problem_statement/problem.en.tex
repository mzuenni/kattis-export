\problemname{Kaldorian Knights}

\illustration{0.28}{codexmanesse}{Painting of a medieval tournament, \href{https://commons.wikimedia.org/wiki/File:Codex_Manesse_(Herzog)_von_Anhalt.jpg}{Codex Manesse}.}%
The king of Kaldoria traditionally celebrates his birthday by
inviting the knights of his realm to a big jousting tournament,
and every noble house participates by sending their best knights to win fame and glory.
At the end of the tournament, the king does not only choose a
winner but ranks all $n$ knights from worst to best.

The number of knights belonging to house $i$ is denoted by $k_i$.
Each knight serves at most one house.
There can be knights who do not serve any house.
The houses are ordered by their influence in the kingdom (with the first one being the most influential).
If the $k_1$ knights of the most powerful house take the last $k_1$ places in
the tournament, the house will incite a revolt against king and crown.
The members of the second most influential house are not that powerful.
Even if all its $k_2$ knights end up at the bottom of the ranking, it
would be considered a strong provocation but the house would not be able to start a revolt.
However, if the last $k_1 + k_2$ places are taken by all the
knights of the two most influential houses combined, then the two houses will unite
and start fighting the king.
More generally, if the knights of the $\ell$ most powerful
houses occupy the last $k_1 + k_2 + \dots
+ k_\ell$ places in the tournament, they will unite and incite a revolt.

Of course, a revolt has to be avoided at all cost. Knowing that the
king often chooses the ranking spontaneously and without too much
consideration, the chief mathematician of the crown has been tasked with analysing
how many rankings will \emph{not} lead to a revolt.

\begin{Input}
  The input consists of:
  \begin{itemize}
    \item One line with integers $n$ ($1 \leq n \leq 10^6$) and $h$
      ($0 \leq h \leq 5000$), the number of knights and the
      number of houses.
    \item $h$ lines, with the $i$th line containing an integer $k_i$ ($1 \leq k_i \leq n$),
      denoting the number of knights from house $i$. Note that every
      house is represented by at least one knight.
  \end{itemize}
  It holds that $\sum_{i=1}^h k_i \leq n$.
\end{Input}

\begin{Output}
  Output the number of rankings that do not lead to a revolt. As
  this number can be quite large, it should be output modulo
  $10^9+7$.
\end{Output}
